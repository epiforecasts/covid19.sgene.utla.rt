% Options for packages loaded elsewhere
\PassOptionsToPackage{unicode}{hyperref}
\PassOptionsToPackage{hyphens}{url}
%
\documentclass[
]{article}
\usepackage{lmodern}
\usepackage{amssymb,amsmath}
\usepackage{ifxetex,ifluatex}
\ifnum 0\ifxetex 1\fi\ifluatex 1\fi=0 % if pdftex
  \usepackage[T1]{fontenc}
  \usepackage[utf8]{inputenc}
  \usepackage{textcomp} % provide euro and other symbols
\else % if luatex or xetex
  \usepackage{unicode-math}
  \defaultfontfeatures{Scale=MatchLowercase}
  \defaultfontfeatures[\rmfamily]{Ligatures=TeX,Scale=1}
\fi
% Use upquote if available, for straight quotes in verbatim environments
\IfFileExists{upquote.sty}{\usepackage{upquote}}{}
\IfFileExists{microtype.sty}{% use microtype if available
  \usepackage[]{microtype}
  \UseMicrotypeSet[protrusion]{basicmath} % disable protrusion for tt fonts
}{}
\makeatletter
\@ifundefined{KOMAClassName}{% if non-KOMA class
  \IfFileExists{parskip.sty}{%
    \usepackage{parskip}
  }{% else
    \setlength{\parindent}{0pt}
    \setlength{\parskip}{6pt plus 2pt minus 1pt}}
}{% if KOMA class
  \KOMAoptions{parskip=half}}
\makeatother
\usepackage{xcolor}
\IfFileExists{xurl.sty}{\usepackage{xurl}}{} % add URL line breaks if available
\IfFileExists{bookmark.sty}{\usepackage{bookmark}}{\usepackage{hyperref}}
\hypersetup{
  pdftitle={Population-level association between S-gene target failure and the relationship between cases, hospitalisations and deaths of Covid-19},
  pdfauthor={Sam Abbott, Sebastian Funk on behalf of the CMMID Covid-19 Working Group},
  hidelinks,
  pdfcreator={LaTeX via pandoc}}
\urlstyle{same} % disable monospaced font for URLs
\usepackage[margin=1in]{geometry}
\usepackage{graphicx,grffile}
\makeatletter
\def\maxwidth{\ifdim\Gin@nat@width>\linewidth\linewidth\else\Gin@nat@width\fi}
\def\maxheight{\ifdim\Gin@nat@height>\textheight\textheight\else\Gin@nat@height\fi}
\makeatother
% Scale images if necessary, so that they will not overflow the page
% margins by default, and it is still possible to overwrite the defaults
% using explicit options in \includegraphics[width, height, ...]{}
\setkeys{Gin}{width=\maxwidth,height=\maxheight,keepaspectratio}
% Set default figure placement to htbp
\makeatletter
\def\fps@figure{htbp}
\makeatother
\setlength{\emergencystretch}{3em} % prevent overfull lines
\providecommand{\tightlist}{%
  \setlength{\itemsep}{0pt}\setlength{\parskip}{0pt}}
\setcounter{secnumdepth}{-\maxdimen} % remove section numbering
\usepackage{float}

\title{Population-level association between S-gene target failure and the
relationship between cases, hospitalisations and deaths of Covid-19}
\usepackage{etoolbox}
\makeatletter
\providecommand{\subtitle}[1]{% add subtitle to \maketitle
  \apptocmd{\@title}{\par {\large #1 \par}}{}{}
}
\makeatother
\subtitle{Work in progress - not peer reviewed}
\author{Sam Abbott, Sebastian Funk on behalf of the CMMID Covid-19 Working Group}
\date{12 January, 2021}

\begin{document}
\maketitle

For correspondence:
\href{mailto:sebastian.funk@lshtm.ac.uk}{\nolinkurl{sebastian.funk@lshtm.ac.uk}}

\hypertarget{abstract}{%
\subsection{Abstract}\label{abstract}}

\textbf{Background:} Individual level data indicates S-gene target
failure may be associated with increased case fatality rates, increased
hospitalisation rates, and increased hospitalisation fatality rates.
Whilst an individual level data approach represents the gold standard
for observational analysis population level analysis may be helpful to
triangulate findings, especially when individual data sources are
confidential, or only partially representative. In this analysis, we use
multiple approaches to evaluate public population level data for
evidence of an association between S-gene target failure and severity
measures.

\textbf{Method:} We explored the association between the proportion of
samples that were S-gene negative and the case fatality rate,
hospitalisation rate, and hospitalisation fatality rate of Covid-19
aggregated at the UTLA and NHS region level. Two approaches were used
with the first assuming a fixed lag between primary and secondary
observations with the lag optimised using the Pearson's correlation
coefficient. The second approach assumed that the secondary observations
could be estimated using a convolution of primary observations
multiplied by some scaling factor. For the fixed lag analysis we
investigated both additive and negative effects of being S-gene negative
and for the convolution approach we explored delays between observations
that varied spatially. We present both univariate and multivariate
estimates with the latter adjusted for spatial and temporal variation.

\textbf{Results:} TODO: summarise results.

\textbf{Conclusions:} TODO: Results and what it means. Our convolution
regression approach may be useful for others where individual data is
not available or subject to biases. It has been implemented in a
generalised framework with all code made publicly available.

\hypertarget{method}{%
\section{Method}\label{method}}

\hypertarget{data}{%
\subsection{Data}\label{data}}

We used 4 main sources of data: test positive Covid-19 notifications by
UTLA,\textsuperscript{{[}1{]}} hospitalisations with Covid-19 by
UTLA,\textsuperscript{{[}2{]}} deaths linked to Covid-19 notification
within 28 days of notification,\textsuperscript{{[}1{]}} and S-gene
status from PCR tests by local authority provided by Public Health
England (PHE)\textsuperscript{{[}3{]}}. We aggregated the data at the
weekly, or daily, level and restricted the analysis to the period
beginning Monday, 5 October.

\hypertarget{statistical-analysis}{%
\subsection{Statistical analysis}\label{statistical-analysis}}

We calculated the weekly proportion of positive tests that were S-gene
negative over time by local authority and NHS region. We estimated the
proportion of tests that were S-gene positive by date of infection
shifting all estimates back by a week. We then conducted two analyses.
In the first analysis lags between cases, admissions and deaths were
estimated by maximising Pearson's correlation coefficient and all data
was then adjusted using these lags to date of infection. In the second
analysis the delay between observations (for example deaths and cases)
was assumed to be log normal with this then being estimated in model
either globally (``global convolution'') or locally using a random
effect (``local convolution''). All analyses were repeated at NHS region
and upper-tier local authority (UTLA) scales. Futher details of each
analysis are given in the following sections.

\hypertarget{fixed-lag-analysis}{%
\subsubsection{Fixed lag analysis}\label{fixed-lag-analysis}}

We assumed that the observed number of Covid-19 admissions/deaths
(\(D_{i,t}\)) within 28 days by date of infection were a function of
Covid-19 notifications/admissions (\(C_{i,t}\)) by date of infection
scaled by the case fatality rate of S-gene positive cases (\(c^+\)) and
S-gene negative cases (\(c^-\)),

\[D_{i,t} \sim \mathrm{NB}\left(c^{+}\left(1-f_{it}\right)C_{i,t} + c^{-}f_{it}C_{i,t} + \epsilon,  \phi \right)\]
where \(i\) indicates UTLA or NHS region, \(t\) week of infection,
\(\epsilon\) is an error term that accounts for imported
deaths/admissions not linked to local cases/admissions, and \(f_{it}\)
is the fraction of cases that were found to be S-gene negative by UTLA
each week. The case fatality rate (or hospitalisation-fatality rate /
case-hospitalisation rate, respectively) of S-gene negative cases then
assumed was then assumed to be a function of static local, and temporal
variation.

\[c^{+} = \mathrm{logit}^{-1}\left(\gamma_{i,t}\right)\] where
\(\gamma_{i,t}\) is either a UTLA-level intercept
\(\gamma_{i,t}\equiv \delta_i\) corresponding to the baseline case
fatality rate per UTLA, or a temporal intercept
\(\gamma_{i,t}\equiv \theta_t\) corresponding to the baseline case
fatality rate over time, with variation over time incorporated using a
thin plate spline. In other words, we stratify the data set either by
UTLA or by week and determine whether differences in the associations
between cases, admissions and deaths are explained by changes in
proportion of cases that are SGTF over time and space, respectively.

The S-gene negative case fatality rate was then assumed to be related to
the S-gene positive case fatality rate via a multiplicative
relationship,

\[c^{-} = \alpha c^{+}\]

or an additive relationship

\[c^{-} = \alpha + c^{+}\]

where \(\alpha\) represents either the multiplicative change in case
fatality rate or the additive change. These alternative
parameterisations represent either a population wide effect for the
former parameterisation or a subpopulation effect in the latter
parameterisation.

\hypertarget{convolution-analysis}{%
\subsubsection{Convolution analysis}\label{convolution-analysis}}

We assumed that the observed number of Covid-19 admissions/deaths
(\(D_{i,t}\)) by date of report were a function of Covid-19
notifications/admissions (\(C_{i,t}\)) by date of report, convolved by a
log normal delay, and scaled by a rate (which when using cases and
deaths is the case fatality rate),

\[D_{i,t} \sim \mathrm{NB}\left(c_i\sum_{\tau = 0}^{30} \xi_{i, \tau} C_{i, t-\tau},  \phi \right)\]

where \(i\) indicates UTLA or NHS region, \(t\) day of report,
\(\xi_{i, \tau}\) is the probability mass function of a log normal
distribution and may be either be static across all locations or vary by
location, and \(\tau\) indexes days prior to \(t\). \(c_i\) is the
location specific case fatality rate (or hospitalisation-fatality rate /
case-hospitalisation rate, respectively). \(c_i\) is then estimated
using,

\[c_i = \mathrm{logit}^{-1}\left(\alpha f_{it} + \gamma_{i,t}\right)\]
where, as for the fixed lag analysis, \(\gamma_{i,t}\) is either a
UTLA-level intercept \(\gamma_{i,t}\equiv \delta_i\) corresponding to
the baseline case fatality rate per UTLA, or a temporal intercept
\(\gamma_{i,t}\equiv \theta_t\) corresponding to the baseline case
fatality rate over time, with time modelled using a thin plate spline.

All models were implemented using the
\texttt{brms}\textsuperscript{{[}4{]}} package in \texttt{R}. All code
required to reproduce this analysis is available from
\url{https://github.com/epiforecasts/covid19.sgene.utla.rt/}.

\hypertarget{results}{%
\section{Results}\label{results}}

TODO: Discuss univariate findings and evidence in scatter plot. Start at
UTLA level and summarise difference at NHS region (if any). Summarise
differences across models.

TODO: Discuss multivariate findings starting with UTLA and summarise
difference at NHS region level (if any). Discuss impact of adjusting for
covariates (and potentially subsets of covariates with these needing to
be added to the SI). Discuss impact of convolution vs lag.

\begin{figure}
\centering
\includegraphics{severity-report_files/figure-latex/scatter-1.pdf}
\caption{Proportion with S gene dropped compared to the adjusted
severity rates each week beginning Monday the 5th of October by NHS
region and upper-tier local authority (UTLA). Each point represents one
NHS region or UTLA and one week, with the size of the point given by the
number of PCR tests.}
\end{figure}

\begin{landscape}\begin{table}

\caption{\label{tab:univariate-effects}Estimated unadjusted effect of S-gene negativity on severity rates (median with with 95\% credible intervals) for the both additive and multiplicative assumptions across spatial aggregations and delay approaches considered. In additive models the effect can be interpreted as a direct change in the rate related to S-gene negativity whilst in multiplicative model the effect can be interpreted as a scaling of the S-gene positive rate.}
\centering
\begin{tabular}[t]{llllll}
\toprule
Method & Aggregation & Effect type & Case fatality rate & Case hospitalisation rate & Hospitalisation fatality rate\\
\midrule
Global lag & NHS region & Additive & 0.02 (0.015, 0.024) & 0.03 (0.016, 0.044) & 0.028 (0.01, 0.046)\\
Global lag & UTLA & Additive & 0.013 (0.011, 0.014) & 0.028 (0.023, 0.032) & 0.054 (0.041, 0.068)\\
Global lag & NHS region & Multiplicative & 2.4 (1.8, 3.3) & 2 (1.5, 2.6) & 1.3 (1.2, 1.5)\\
Global lag & UTLA & Multiplicative & 2 (1.8, 2.2) & 1.6 (1.5, 1.7) & 1.4 (1.3, 1.5)\\
Global convolution & NHS region & Multiplicative & 2.2 (2, 2.4) & 1.4 (1.3, 1.6) & 1.4 (1.2, 1.5)\\
\addlinespace
Global convolution & UTLA & Multiplicative & 1.8 (1.7, 1.9) & 1.4 (1.3, 1.5) & 1.4 (1.2, 1.5)\\
Local convolution & NHS region & Multiplicative & 2.2 (2, 2.4) & 1.5 (1.3, 1.7) & 1.3 (1.1, 1.4)\\
\bottomrule
\end{tabular}
\end{table}
\end{landscape}

\begin{landscape}\begin{table}

\caption{\label{tab:multivariate-effects}Estimated adjusted effect of S-gene negativity on severity rates (median with with 95\% credible intervals) for the both additive and multiplicative assumptions across spatial aggregations and delay approaches considered. In additive models the effect can be interpreted as a direct change in the rate related to S-gene negativity whilst in multiplicative model the effect can be interpreted as a scaling of the S-gene positive rate. Effect estimates are adjusted for location variability and for current case/hospital admissions.}
\centering
\begin{tabular}[t]{llllll}
\toprule
Method & Aggregation & Effect type & Case fatality rate & Case hospitalisation rate & Hospitalisation fatality rate\\
\midrule
Global lag & NHS region & Additive & 0.021 (0.018, 0.025) & 0.041 (0.029, 0.051) & 0.008 (-0.011, 0.026)\\
Global lag & UTLA & Additive & 0.016 (0.014, 0.017) & 0.043 (0.04, 0.047) & 0.049 (0.034, 0.064)\\
Global lag & NHS region & Multiplicative & 2.8 (2.3, 3.4) & 2.1 (1.8, 2.5) & 1.1 (0.96, 1.3)\\
Global lag & UTLA & Multiplicative & 2.6 (2.3, 2.8) & 2.1 (1.9, 2.2) & 1.4 (1.3, 1.5)\\
Global convolution & NHS region & Multiplicative & 2.4 (2.1, 2.6) & 1.6 (1.5, 1.8) & 0.96 (0.84, 1.1)\\
\addlinespace
Global convolution & UTLA & Multiplicative & 2 (1.8, 2.1) & 1.7 (1.6, 1.8) & 1.2 (1.1, 1.3)\\
Local convolution & NHS region & Multiplicative & 2.4 (2.1, 2.6) & 1.6 (1.5, 1.8) & 0.95 (0.83, 1.1)\\
\bottomrule
\end{tabular}
\end{table}
\end{landscape}

\hypertarget{discussion}{%
\section{Discussion}\label{discussion}}

We studied the relationship between SGTF (as a proxy for the new variant
of concern) and the association between Covid-19 cases, hospitalisations
and deaths adjusted using multiple approaches.

TODO: Summarise convolution findings (univariate and multivariate, by
UTLA and NHS region). TODO: Summarise lag findings (univariate and
multivariate, by UTLA and NHS region). TODO: Summarise differences
across approaches (spatial scale, and model).

Our estimates for the association between SGTF and the case fatality
rate were comparable to those from individual based approaches
but\ldots{}

TODO: Compare to lshtm + exeters + imperials case control and survival
analysis of case fatality rate. TODO: Compare to other studies looking
at hospitalisation rate + hospitalisation fataility rate.

Our results are indicative only as they make use of aggregated data that
is subject to a large range of confounders. However, they may act as
useful support for other, individual level approaches, and potentially
may be generalised to scenarios where individual level data is not
available. Whilst we adjusted for location and temporal variation we
could not adjust for multiple confounders such as the age of
cases/admission due to the lack of public data on the age of Covid-19
cases/admissions by NHS region or UTLA. Our inclusion of temporal
variation as a confounder may also compete with the effect of S-gene
target failure causing bias in our multivariate estimates. Lastly, we
fitted the model only point estimates of the proportion of SGTF observed
in every UTLA per week. Because of this, uncertainty in our regression
coefficients are underestimated, and probably considerably so. Finally,
SGTF uniquely identify the novel variant and therefore our analysis may
be biased by the inclusion of other variants. Our results should be used
to triangulate the effect of novel variant rather than as standalone
evidence.

TODO: Improve limitations. TODO: Conclusions. What does it mean. Why is
this a good way of doing it. What can we do next. Method
generalisability. Wrap up.

\hypertarget{references}{%
\section{References}\label{references}}

\hypertarget{refs}{}
\leavevmode\hypertarget{ref-ukgov}{}%
1. \emph{Coronavirus (covid-19) in the uk}. (2021).
\url{https://coronavirus.data.gov.uk/details/healthcare}.

\leavevmode\hypertarget{ref-covidnhsdata}{}%
2. Meakin, S., Abbott, S., \& Funk, S. (2021). \emph{NHS trust level
covid-19 data aggregated to a range of spatial scales}.
\url{https://doi.org/10.5281/zenodo.4447465}

\leavevmode\hypertarget{ref-phe}{}%
3. England, P. H. (2020). \emph{Investigation of novel sars-cov-2
variant: Variant of concern 202012/01.}
\url{https://www.gov.uk/government/publications/investigation-of-novel-sars-cov-2-variant-variant-of-concern-20201201}.

\leavevmode\hypertarget{ref-brms}{}%
4. Bürkner, P.-C. (2018). Advanced Bayesian multilevel modeling with the
R package brms. \emph{The R Journal}, \emph{10}(1), 395--411.
\url{https://doi.org/10.32614/RJ-2018-017}

\end{document}
